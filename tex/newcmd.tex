\ifdefined\texrc@newcmd\else
\def\texrc@newcmd{LOADED}

%% Some \def and \newcommand macros

%% It maybe used to a title.
\long\def\uphrule{\newskip\ruledist \ruledist=2pt%
\newskip\ruleneg \ruleneg=\baselineskip \advance\ruleneg by-\ruledist%
\vbox{\hbox to\hsize{\vrule width\hsize height0.6em depth-0.5em}%
\vskip-\ruleneg\hbox to\hsize{\vrule width\hsize height0.55em depth-0.5em}}}
\long\def\downhrule{\newskip\ruledist \ruledist=2pt%
\newskip\ruleneg \ruleneg=\baselineskip \advance\ruleneg by-\ruledist%
\ruledist=.05em\advance\ruleneg by-\ruledist%
\vbox{\hbox to\hsize{\vrule width\hsize height0.55em depth-0.5em}%
\vskip-\ruleneg\hbox to\hsize{\vrule width\hsize height0.6em depth-0.5em}}}

% generate a box that width equal to zero, but height and
% depth can be adjusted by para2,3
\def\adbox#1#2#3{\newdimen\d@p \d@p=\dp#1%
    \newdimen\h@t \h@t=\ht#1%
    \advance\d@p by #3%
    \advance\h@t by #2%
    \hbox{\vrule width0pt depth\d@p height\h@t}
}
%{{{ typeset for practicing letters
% get the length of token
\newcount\ts@length@count
\def\token@length{%
    \ts@length@count=0\relax%
    \futurelet\@char@next\ts@count@check%
}
\def\ts@count@check{%
    \ifx\@char@next\end%
        \expandafter\length@omit@end%
    \else
        \advance\ts@length@count\@ne%
        \expandafter\cont@check@length%
    \fi%
}
\def\cont@check@length#1{%
    \futurelet\@char@next\ts@count@check%
}
\def\length@omit@end#1{}

% omit let
\def\let@omit@to@end#1{%
    \futurelet#1\@omit@to@end%
}
\def\@omit@to@end{%
    \futurelet\omit@next@char\cont@check%
}
\def\cont@check{%
    \ifx\omit@next@char\end%
        \expandafter\omit@end%
    \else
        \expandafter\cont@check@@%
    \fi%
}
\def\cont@check@@#1{%
    \@omit@to@end%
}
\def\omit@end#1{}

% the width and height of framed box
\newdimen\target@length \target@length=2em
\newdimen\half@target@length \half@target@length\target@length \divide\half@target@length\tw@

% the line width that used in framed box
\newdimen\line@thick \line@thick=.8pt
\newdimen\inner@line@thick \inner@line@thick=.25pt
\newdimen\vertical@offset \vertical@offset\half@target@length \advance\vertical@offset-\line@thick

% temporary char holder
\newbox\temp@char@holder%
\newdimen\calculate@temp
\def\set@temp@holder#1{%
\setbox\temp@char@holder\hbox{#1}%
\calculate@temp\ht\temp@char@holder%
\advance\calculate@temp\dp\temp@char@holder%
\divide\calculate@temp\tw@%
\advance\calculate@temp-\dp\temp@char@holder%
\setbox\temp@char@holder\hbox{\lower\calculate@temp\box\temp@char@holder}%
}

% aftergroup, output this box
\newbox\after@box
\def\set@after@output#1{%
    \setbox\after@box\hbox{\scalebox{#1}{\hbox to\target@length{%
        \vrule width\line@thick depth\half@target@length height\half@target@length\relax%
        \kern-\line@thick%
        \kern\target@length%
        \vrule width\line@thick depth\half@target@length height\half@target@length\relax%
        \kern-\line@thick%
        \kern-\target@length%
        \vrule width\target@length depth-\vertical@offset height\half@target@length\relax%
        \kern-\target@length%
        \vrule width\target@length height-\vertical@offset depth\half@target@length\relax%
        \kern-\target@length%
        {% inner dash lines
            \color{google@blue}%
            \kern.5\line@thick%
            \cleaders\hbox to.2\target@length{\hss\vrule width.14\target@length depth.5\inner@line@thick height.5\inner@line@thick\hss}\hskip\target@length%
            \kern-\target@length%
            \kern-.5\line@thick%
            %
            \kern\half@target@length%
            \vrule width\inner@line@thick depth.47\target@length height-.33\target@length\kern-\inner@line@thick%
            \vrule width\inner@line@thick depth.27\target@length height-.13\target@length\kern-\inner@line@thick%
            \vrule width\inner@line@thick depth.07\target@length height.07\target@length\kern-\inner@line@thick%
            \vrule width\inner@line@thick depth-.13\target@length height.27\target@length\kern-\inner@line@thick%
            \vrule width\inner@line@thick depth-.33\target@length height.47\target@length\kern-\inner@line@thick%
            \kern-\half@target@length%
        }%
        \hfill\box\temp@char@holder\hfill%
    }}}%
}
\newdimen\current@length
\def\ts@zero@cl{\global\current@\length=0pt\relax}
\def\ts@newline{\global\current@length\wd\after@box\relax\break%
\parindent=0pt\relax\vskip-2\line@thick\relax\leavevmode}
\def\ts@adv@cl{\global\advance\current@length\wd\after@box\relax}
\def\after@output{%
    \expandafter\set@after@output\expandafter{\the\scale@reg}%
    \aftergroup{%
    \ts@adv@cl%
    \ifnum\current@length>\hsize%
        \ts@newline%
    \fi%
    \box\after@box%
    %\ \settowidth\temp@length{\ }\kern-\temp@length%
    }%
}

\newtoks\cjk@exec
\newtoks\scale@reg
\newskip\holder@one
\newskip\holder@two
\newif\if@in@math
\newif\if@isnot@macro
\newcommand\btypeset[2][\relax]{%
    \@btypeset{#1}{#2}}
\def\@btypeset#1#2{%
    \scale@reg{#2}%
    \cjk@exec{#1}%
    \@in@mathfalse%
    \current@length=0pt%
    \holder@one=\baselineskip%
    \holder@two=\lineskip%
    \baselineskip=0pt%
    \lineskip=0pt%
    \bgroup%
    \aftergroup{\bgroup\parindent=0pt\relax\leavevmode}%
    \ts@future@char%
}
\def\future@def#1#2#3{\def#1{#3}#2#3}
\def\ts@future@char{%
    \future@def\ts@next@char\ts@output%
}
\def\etypeset{%
    \aftergroup{\hfill\break\vskip-.2\line@thick\relax\egroup}%
    \egroup%
    \baselineskip=\holder@one%
    \lineskip=\holder@two%
}
\def\ts@continue#1{%
    \ts@future@char%
}
\def\ts@continue@macro#1#2{%
    \ts@future@char%
}
%% very strange when using cjk characters
\def\ts@continue@cjk#1#2#3{%
    \ts@future@char%
}
{%
    \catcode`A=11\relax\global\let\eleven@cat=A%
    \catcode`\$=3\relax\global\let\math@cat=$%
    \global\let\math@cat=$%
}
\let\ts@exec@choice\relax%
\def\ts@output{%
    \global\@isnot@macrotrue%
    \expandafter%
    \ifx\ts@next@char\etypeset%
        \relax%
        \let\ts@exec@choice\relax%
    \else
        \expandafter\let@omit@to@end\expandafter\first@token@@\ts@next@char\end%
        \expandafter\let@omit@to@end\expandafter\first@token\first@token@@\end%
        %%%% check whether \ts@next@char is a char macro or a primitive macro
        \expandafter\token@length\ts@next@char\end%
        \ifnum\ts@length@count>1
            \global\@isnot@macrofalse
        \fi
        \ifx\first@token\CJK@@@%
            \set@temp@holder{\the\cjk@exec\ts@next@char}%
            \after@output%
            \let\ts@exec@choice\ts@continue@cjk%
        \else%
            \if@isnot@macro%
                \ifcat\ts@next@char\eleven@cat%
                    \if@in@math%
                        \set@temp@holder{$\ts@next@char$}%
                        \after@output%
                    \else
                        \set@temp@holder{\ts@next@char}%
                        \after@output%
                    \fi
                \else%
                    \ifcat\ts@next@char\math@cat%
                        \if@in@math%
                            \global\@in@mathfalse%
                        \else%
                            \global\@in@mathtrue%
                        \fi%
                    \fi%
                \fi%
                \let\ts@exec@choice\ts@continue%
            \else
                \if@in@math%
                    \set@temp@holder{$\ts@next@char$}%
                    \after@output%
                \else
                    \set@temp@holder{\ts@next@char}%
                    \after@output%
                \fi%
                \let\ts@exec@choice\ts@continue@macro%
            \fi
        \fi%
    \fi%
    \ts@exec@choice%
}
%}}} end typeset
\fi
